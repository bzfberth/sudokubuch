\chapter{Baumsuchalgorithmen: Vom Modell zur Lösung}
\label{chap_technik}

\section{Propagierung}

Wie gehen wir intuitiv vor, wenn wir Sudokus von Hand lösen? Nun, wir
betrachten die Information, die uns bereits bekannt ist und schauen,
ob wir mithilfe der gegebenen Regeln weitere Information herleiten
können. In der Mathematik bezeichnet man ein solches Vorgehen als
\emph{Propagierung}.

Wenn wir zum Beispiel in einem Feld einer Spalte bereits eine Zahl vorgegeben
haben, dann wissen wir, dass diese Zahl in keinem anderen Feld
derselben Spalte stehen darf. Sind in einer Spalte alle bis ein Feld
mit Zahlen belegt, dann wissen wir,d ass in das verbleibende Feld die
verbleibende Zahl gehört.

Das entspricht genau den zwei Propagierungschritten, die man bei
Gleichungen der Form ``Summe von Binärvariablen gleich 1''.
Die Basics: Einer 1 $\leadsto$ alle 0, alle außer einer 0 $\leadsto$ 1

\section{Probing}
was passiert, wenn...? Konflikte, 1 stärker als 0

\section{Branching}
Fortsetzung des Probing, systematisch Suchen, Baumstruktur,
Komplexität

\section{Propagierung für allgemeine Ungleichungen}
% eventuell als Vertiefung in anderes Kapitel?
Min-max-Aktivitäten. Wahrscheinlich auf Ganzzahligkeit beschränken.

\section{Gültige Bedingungen ableiten}
Knapsack-Cover
%%% Local Variables: 
%%% mode: latex
%%% TeX-master: "main"
%%% End: 
