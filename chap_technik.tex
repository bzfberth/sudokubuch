\chapter{Baumsuchalgorithmen: Vom Modell zur Lösung}
\label{chap_technik}

\section{Propagierung}
\label{sect_propagierung}
Wie gehen wir intuitiv vor, wenn wir Sudokus von Hand lösen? Nun, wir
betrachten die Information, die uns bereits bekannt ist und schauen,
ob wir mithilfe der gegebenen Regeln weitere Information herleiten
können. In der Mathematik bezeichnet man ein solches Vorgehen als
\emph{Propagierung}.

Wenn wir zum Beispiel in einem Feld einer Spalte bereits eine Zahl vorgegeben
haben, dann wissen wir, dass diese Zahl in keinem anderen Feld
derselben Spalte stehen darf. Sind in einer Spalte alle bis ein Feld
mit Zahlen belegt, dann wissen wir, dass in das verbleibende Feld die
verbleibende Zahl gehört.

Das entspricht genau den zwei Propagierungschritten, die man bei
Gleichungen der Form ``Summe von Binärvariablen gleich 1'' zur Verfügung hat.
Ist eine der Variablen bereits auf 1 fixiert, so wissen wir, dass wir alle anderen Variablen auf 0 fixieren können. Sind alle bis auf eine Variable auf 0 fixiert, so können wir die letzte verbleibende Variable auf 1 fixieren.
Für Gleichungen mit beliebigen Koeffizienten und Variablen mit einem allgemeinerem Wertebereich kann man sich natürlich noch deutlich involviertere Propagierungsverfahren überlegen. Doch dazu später mehr. 

Hat man auf diese Art und Weise mindestens eine weitere Fixierung gefunden, kann man dass ganze Verfahren erneut anwenden. Runde um Runde sozusagen. Bis man keine weiteren Fixierungen mehr findet. Im besten Fall liegt das daran, dass man das Rätsel bereits gelöst hat, also alle Variablen fixiet wurden. Ist das nicht der Fall, muss man zu komplexeren Verfahren greifen, mit denen wir uns in Kapitel~\ref{sect_probing} und~\ref{sect_branching} beschäftigen werden.
 
Wenn man so möchte, ist Propagierung das rein logische Ableiten korrekter Schlussfolgerungen aus gegebenen Voraussetzungen. Diese Variablenfixierungen kenne ich -- also müssen folgende ebenfalls gelten. Die Propagierung ist unsere Komponente für einen Algorithmus, der Sudokus, oder allgemeinere ganzzahlige Gleichungssysteme, löst.

\section{Probing}
\label{sect_probing}
Was wäre, wenn? Diese Frage geistert uns meist im Kopf herum, wenn wir vor zwei Alternativen gestellt werden. Passiert in beiden Fällen das gleiche? Oder genau etwas entgegengesetztes? Ist eine der beiden Alternativen vielleicht gar nicht möglich? Egal, um welche Frage es konkret geht, wenn man sich entscheiden muss, fängt man an zu grübeln und geht im Kopf beide Varianten durch, um festzustellen, was sie für Konsequenzen haben.

Eine 0-1-Variable stellt uns ebenso vor eine Entscheidung. Am Ende wird in der Lösung des Gleichungssystems einer der beiden Werte angenommen und der andere eben nicht. Das Durchspielen beider Varianten bezeichnet man in der Mathematischen Optimierung als \emph{Probing}. Man fixiert also die Variable vorübergehend auf 0 und lässt Propagierungsverfahren laufen, siehe Kapitel~\ref{sect_propagierung}. Alle sich ergebenden Variablenfixierungen merkt man sich. Dann fixiert man dieselbe Variable vorübergehend auf 1, wendet wieder die Propagierung an und merkt sich das Ergebnis. Nun vergleicht man. 

Nehmen wir einmal an, dass Fixieren einer Variable $x_{i,j,k}$ sowohl auf 0 als auch auf 1, hat dazu geführt, dass in beiden eine andere Variable jeweils auf 0 fixiert wurde. Was bedeutet das? Offensichtlich ist es unabhängig davon, welchen der beiden Werte 0 oder 1 $x_{i,j,k}$ annimmt, es folgt auf jeden Fall, dass die andere Variable auf 0 fixiert werden muss. Also können wir sie bereits jetzt auf Null fixieren, ohne bereits den korrekten Wert für  $x_{i,j,k}$ zu kennen. Einen der beiden Werte 0 oder 1 wird $x_{i,j,k}$ ja in der finalen Lösung annehmen. Und in beiden Fällen muss unsere betrachtete Variable eben 0 sein, diese Erkenntnis haben wir durch Ausprobieren, durchs ``Probing'' erhalten. 

Nun kann es auch passieren, dass eine der beiden Richtungen beim Probing eine ``Sackgasse'' ist.


\section{Branching}

\label{sect_branching}
Fortsetzung des Probing, systematisch Suchen, Baumstruktur,
Komplexität

\section{Propagierung für allgemeine Ungleichungen}
% eventuell als Vertiefung in anderes Kapitel?
Min-max-Aktivitäten. Wahrscheinlich auf Ganzzahligkeit beschränken.

\section{Gültige Bedingungen ableiten}
Knapsack-Cover
%%% Local Variables: 
%%% mode: latex
%%% TeX-master: "main"
%%% End: 
